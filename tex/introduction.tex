\section{Introduction}

Ant colonies regulate their foraging behavior through a collective decision making process; when ants forage, no individual ant operates with complete information about the terrain they are exploring. This contrasts with traditional human approaches decision-making, which typically centralize information, process it, then redistribute instructions. Consider, for example, traffic-aware navigation tools such as Waze or Google Maps; the distribution of traffic distribution across a geographic region is collected from users, centrally processed, and then routing instructions are redistributed to individual users. In contrast, collective intelligence on the level of the ant colony emerges from parallel execution of a simple set of individual pheromone deposit and response behaviors; among other feats, foraging ants will tend to choose the shortest path between nest and food and to selectively exploit the richest of an array of food sources.

My task was to extend mathematical models of ant foraging, which are well-developed on flat surfaces, to uneven terrains. It is well known that, as a collective, ants can optimize the foraging path they travel between nest and food. On flat terrain, a clear ``best'' path exists: the shortest-distance foraging path, the most energy-efficient path, and the quickest-trip path are all identical. On uneven terrains, however, this is no longer necessarily the case and the question of which trade-offs ants make -- and how they make them -- is of great interest. After surveying existing models of ant foraging behavior on flat terrain and individual ant behavior on inclined surfaces, I designed and numerically evaluated a differential equations-based model of the foraging behavior of ants over uneven terrain.

My model will see continued use, enabling the Swarm Lab to work out the rules by which real ants act on uneven terrain in a foraging context by simulating hypothesized behaviors and assessing the resulting foraging path predictions in comparison with upcoming in vivo ant foraging experiments. Ultimately, research into the collective intelligence of insects translates directly to technological applications; for example, such research has been leveraged in swarm robotics projects, such as NASA's ant-inspired ``Swarmies'' that may one day harvest resources for Martian colonies and distributed traffic management schemes (especially in relation to autonomous vehicles).